%%%%%%%%%%%%%%%% Packages utilisés %%%%%%%%%%%%%%%%

%%%%% Encodage, Police et Langue %%%%%
\usepackage[latin1]{inputenc}	% Définit l'encodage du fichier
\usepackage[T1]{fontenc}	    % Charge les polices d'affichage adaptées
\usepackage{lmodern}		    % Charge les polices vectorielles
\usepackage[french]{babel}	    % Module de langue française

%%%%% Graphiques et mise en forme avancée %%%%%
\usepackage{graphicx}       % Permet de charger des images
\usepackage{xcolor}		    % Module permettant de gérer des couleurs
\usepackage{array}		    % Permet d'utiliser des tableaux
\usepackage{framed}		    % Permet d'ajouter des "cadres"
\usepackage{hyperref}       % Rend tous les liens internes cliquable
\usepackage[all]{hypcap}	% Liens pointent sur figures et non titres
\usepackage{vmargin}		% Permet de redéfinir les marges

% Supprime la bordure rouge des liens cliquables
\hypersetup{pdfborder={0 0 0 [3 3]}}

% \setmarginsrb{1}{2}{3}{4}{5}{6}{7}{8}
% 1 est la marge gauche
% 2 est la marge en haut
% 3 est la marge droite
% 4 est la marge en bas
% 5 fixe l'espace à gauche
% 6 fixe l'entête
% 7 fixe l'espace à droite
% 8 fixe le pied de page
%   _ _ _ _ _ _ _
%  |  _ _ 2 _ _  |
%  | |    6    | |
%  | |         | |
%  | |         | |
%  | |         | |
%  |1|5 Texte 7|3|
%  | |         | |
%  | |         | |
%  | |         | |
%  | |_ _ 8 _ _| |
%  |_ _ _ 4 _ _ _|
%
\newcommand{\marge}{2cm}		% Marge générale du document
\setmarginsrb{\marge}{\marge}{\marge}{\marge}
{0cm}{3cm}{0cm}{3cm}			% Garde un écart pour le numéro de page & titres


%%%%% Packages scientifiques %%%%%
\usepackage{tikz,tkz-tab}		% Tableaux de variations
\usepackage{amsmath}			% Fonctions mathématiques évoluées
\usepackage{amssymb}			% Symboles mathématiques évoluées
\usepackage{sistyle}			% Valeurs avec unités du Système Inter
\usepackage[overload]{empheq}	% Utile pour des équations avec accolades

%%%%% Packages développeurs %%%%%
\usepackage{listings}			% Codes sources formatés
\usepackage{verbatim}			% Textes non-interprétés

% Définition du style d'affichage des codes sources insérés
\lstset{ %
	%language=C,
	basicstyle=\footnotesize,
	numbers=left,
	numberstyle=\tiny\color{gray},
	stepnumber=1,
	numbersep=5pt,
	backgroundcolor=\color{white},
	frame=lines,
	rulecolor=\color{black},
	captionpos=tb,
	breaklines=true,
	title=\lstname,
	showstringspaces=false,
	keywordstyle=\color{blue},
	commentstyle=\color{gray},
	stringstyle=\color{red},
}

%%%%% Packages divers %%%%%
\usepackage{marvosym}			% Permet d'utiliser \EUR => €


%%%%%%%%%%%%%%%% Commandes personnalisées %%%%%%%%%%%%%%%%

%%%%% Commandes mathématiques %%%%%

% Réalise une exponentielle de la forme exp[exposant]
\newcommand{\e}[1]
{\mathrm{~exp}\left[#1\right]}

% Imprime le d de dérivée correctement (pas comme une variable d)
\renewcommand{\d}{\mathrm{d}}

% Met des accolades sur le côté d'une liste d'éléments
\newcommand{\accolades}[1]
{\left \{ \begin{array}{l c r} #1 \end{array} \right. }

% Met l'équation à gauche et une description à gauche
\newcommand{\EquationCarac}[2]
{
	\medskip
	\noindent\begin{minipage}{0.4\linewidth}
		#1
	\end{minipage}
	\begin{minipage}{0.6\linewidth}%
		\begin{leftbar}
			#2
		\end{leftbar}
	\end{minipage}
	\medskip
}

%%%%% Commandes développeurs %%%%%
% Charge le fichier texte et l'affiche de manière brute (sans interprétation)
\newcommand{\cmd}[1]
{\begin{leftbar}\verbatiminput{#1}\end{leftbar}}


%%%%% Commandes pratiques diverses %%%%%

% Réalise un symbole Euro (sans se soucier du mode math ou non)
\newcommand{\euro}
{\text{\EUR}}

% Crée un hyperlien numérotée vers la figure 
\newcommand{\reffig}[1]
{\textit{\hyperref[fig:#1]{Figure \ref*{fig:#1}}}}

% Charge la figure donnée en premier paramètre et qui doit être dans le 
% dossier Schemas. Le second paramètre défini le titre.
\newcommand{\fig}[2]
{\begin{figure}[htbp]
	\centering\includegraphics{Schemas/#1}\caption{#2}\label{fig:#1}
\end{figure}}



