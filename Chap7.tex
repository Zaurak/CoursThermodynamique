\chapter{Diffusion thermique}

\section{Loi de Fourier}

	\subsection{Introduction}
	
	\begin{itemize}
		\item La chaleur va du chaud vers le froid.
		\item Consid�rons une surface $S$ entre $t$ et $t + \d t$, un transfert thermique $ \delta Q$ a lieu.
		\item On pose 
			\[
				\boxed{\delta Q = \Phi \d t}
			\]
			O� $\Phi = $ flux thermique � travers $S$
	\end{itemize}
	
	\paragraph{N.B.}
	\begin{itemize}
		\item $\big[ \Phi\big] = J.s^{-1} = W$
		\item Sur un mur, $\Phi$ est la puissance totale traversant le mur d'un point de vue macroscopique.
		\item Au niveau microscopique, on d�finit $\vec{j_Q}$ tel que
			\[
				\boxed{\Phi = \iint_{P \in S} \vec{j_Q}(P) \d \vec{S_P}}
			\]
			Avec :
			\begin{itemize}
				\item $\vec{j_Q}= $ vecteur densit� de courant de chaleur
				\item $\big[ j_Q\big] = W m^{-2}= $ c'est une puissance surfacique.
			\end{itemize}
	\end{itemize}
	
	\subsection{Loi de Fourier}
	
	\paragraph{\'Enonc�} Au voisinage de l'�quilibre ($\vec{\text{grad}} T$ pas trop grand), on a:
	\[
		\boxed{\vec{j_Q} = - \lambda \vec{\text{grad}} T}
	\]
	Avec :
	\begin{itemize}
		\item $ \vec{j_Q} $ en $Wm^{-2}$
		\item $ \lambda $ en $W m^{-1} K^{-1} $
		\item $T$ en $K$
	\end{itemize}
	
	\paragraph{N.B.}
	\begin{itemize}
		\item $\lambda = $ conductivit� du milieu
			\subitem Si $ \lambda$ grand (ex: m�taux), milieu bon conducteur thermique
			\subitem $\lambda$ petit (ex: air, laine de verre, ...), milieu mauvais.

	\end{itemize}
