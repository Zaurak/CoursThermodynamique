\documentclass[a4paper, 11pt]{book}

% Inclue tout les packages n�cessaires/utiles
%%%%%%%%%%%%%%%% Packages utilisés %%%%%%%%%%%%%%%%

%%%%% Encodage, Police et Langue %%%%%
\usepackage[latin1]{inputenc}	% Définit l'encodage du fichier
\usepackage[T1]{fontenc}	    % Charge les polices d'affichage adaptées
\usepackage{lmodern}		    % Charge les polices vectorielles
\usepackage[french]{babel}	    % Module de langue française

%%%%% Graphiques et mise en forme avancée %%%%%
\usepackage{graphicx}       % Permet de charger des images
\usepackage{xcolor}		    % Module permettant de gérer des couleurs
\usepackage{array}		    % Permet d'utiliser des tableaux
\usepackage{framed}		    % Permet d'ajouter des "cadres"
\usepackage{hyperref}       % Rend tous les liens internes cliquable
\usepackage[all]{hypcap}	% Liens pointent sur figures et non titres
\usepackage{vmargin}		% Permet de redéfinir les marges

% Supprime la bordure rouge des liens cliquables
\hypersetup{pdfborder={0 0 0 [3 3]}}

% \setmarginsrb{1}{2}{3}{4}{5}{6}{7}{8}
% 1 est la marge gauche
% 2 est la marge en haut
% 3 est la marge droite
% 4 est la marge en bas
% 5 fixe l'espace à gauche
% 6 fixe l'entête
% 7 fixe l'espace à droite
% 8 fixe le pied de page
%   _ _ _ _ _ _ _
%  |  _ _ 2 _ _  |
%  | |    6    | |
%  | |         | |
%  | |         | |
%  | |         | |
%  |1|5 Texte 7|3|
%  | |         | |
%  | |         | |
%  | |         | |
%  | |_ _ 8 _ _| |
%  |_ _ _ 4 _ _ _|
%
\newcommand{\marge}{2cm}		% Marge générale du document
\setmarginsrb{\marge}{\marge}{\marge}{\marge}
{0cm}{3cm}{0cm}{3cm}			% Garde un écart pour le numéro de page & titres


%%%%% Packages scientifiques %%%%%
\usepackage{tikz,tkz-tab}		% Tableaux de variations
\usepackage{amsmath}			% Fonctions mathématiques évoluées
\usepackage{amssymb}			% Symboles mathématiques évoluées
\usepackage{sistyle}			% Valeurs avec unités du Système Inter
\usepackage[overload]{empheq}	% Utile pour des équations avec accolades

%%%%% Packages développeurs %%%%%
\usepackage{listings}			% Codes sources formatés
\usepackage{verbatim}			% Textes non-interprétés

% Définition du style d'affichage des codes sources insérés
\lstset{ %
	%language=C,
	basicstyle=\footnotesize,
	numbers=left,
	numberstyle=\tiny\color{gray},
	stepnumber=1,
	numbersep=5pt,
	backgroundcolor=\color{white},
	frame=lines,
	rulecolor=\color{black},
	captionpos=tb,
	breaklines=true,
	title=\lstname,
	showstringspaces=false,
	keywordstyle=\color{blue},
	commentstyle=\color{gray},
	stringstyle=\color{red},
}

%%%%% Packages divers %%%%%
\usepackage{marvosym}			% Permet d'utiliser \EUR => €


%%%%%%%%%%%%%%%% Commandes personnalisées %%%%%%%%%%%%%%%%

%%%%% Commandes mathématiques %%%%%

% Réalise une exponentielle de la forme exp[exposant]
\newcommand{\e}[1]
{\mathrm{~exp}\left[#1\right]}

% Imprime le d de dérivée correctement (pas comme une variable d)
\renewcommand{\d}{\mathrm{d}}

% Met des accolades sur le côté d'une liste d'éléments
\newcommand{\accolades}[1]
{\left \{ \begin{array}{l c r} #1 \end{array} \right. }

% Met l'équation à gauche et une description à gauche
\newcommand{\EquationCarac}[2]
{
	\medskip
	\noindent\begin{minipage}{0.4\linewidth}
		#1
	\end{minipage}
	\begin{minipage}{0.6\linewidth}%
		\begin{leftbar}
			#2
		\end{leftbar}
	\end{minipage}
	\medskip
}

%%%%% Commandes développeurs %%%%%
% Charge le fichier texte et l'affiche de manière brute (sans interprétation)
\newcommand{\cmd}[1]
{\begin{leftbar}\verbatiminput{#1}\end{leftbar}}


%%%%% Commandes pratiques diverses %%%%%

% Réalise un symbole Euro (sans se soucier du mode math ou non)
\newcommand{\euro}
{\text{\EUR}}

% Crée un hyperlien numérotée vers la figure 
\newcommand{\reffig}[1]
{\textit{\hyperref[fig:#1]{Figure \ref*{fig:#1}}}}

% Charge la figure donnée en premier paramètre et qui doit être dans le 
% dossier Schemas. Le second paramètre défini le titre.
\newcommand{\fig}[2]
{\begin{figure}[htbp]
	\centering\includegraphics{Schemas/#1}\caption{#2}\label{fig:#1}
\end{figure}}





\newcommand{\corrector}
{Quentin \bsc{Marques} - \url{marques@ece.fr}}

\begin{document}
 
\title{Cours de Thermodynamique}
\author{M. Nicolas \bsc{Gaudin}}
				
\date{\today} 
 
\maketitle
\newpage
 
\tableofcontents
\newpage

\frontmatter

\chapter{Pr�ambule}

\paragraph{Origine de ce support de cours}
Ce cours est une retranscription du cours dispens� par M. Nicolas \bsc{Gaudin} pour l'ECE Paris. Ce dernier a donn� son accord pour la libre diffusion du cours aux �l�ves. Le cours a �t� retranscrit par des �l�ves, pour des �l�ves. 

\paragraph{Gare aux erreurs}
Ce cours contient certainement des erreurs ou des notes incompl�tes. Si vous remarquez une telle chose et que vous �tes s�r(e) de votre fait (ou mieux, que M. Gaudin a valid� votre correction), vous pouvez :

\begin{itemize}
    \item \emph{Signaler l'erreur} : Le principal correcteur actuellement capable de mettre � jour le document pour vous peut �tre joint via : \corrector.
    \item \emph{Corriger vous-m�me l'erreur} : Le cours est �crit en \LaTeX et publi� sur GitHub. Vous pouvez, sous r�serve de ma�triser ces deux technologies un minimum, corriger l'erreur et publier le correctif sur GitHub.
\end{itemize}

\paragraph{Remerciements}
Nous tenons � remercier M. \bsc{Gaudin} pour avoir autoris� la diffusion de son cours ainsi que les �l�ves ayant particip� � la r�daction ou la correction de ce support !

Bonne lecture !

\mainmatter

\chapter{Bases de la thermodynamique}

	\section{Introduction}

				\paragraph{Exemple 1}
				Un paquet de chips gonfle en altitude car les mol�cules du gaz int�rieur du paquet prennent plus de place. En effet, il y a moins de pressions � l'ext�rieur de paquet car la pression $ P(z) $ diminue quand la l'altitude $z$ augmente.
				(Voir \reffig{paquet_de_chips}).
				
						\fig{paquet_de_chips}
						{Augmentation de l'altitude sur un paquet de chips}
								
				\paragraph{Exemple 2}
				Dans l'amphith��tre, la temp�rature de la table en bois est diff�rente de celle de la barre en m�tal. En effet, le bois conduit moins bien la chaleur que le m�tal.
				Attention! La temp�rature et l'�nergie sont deux notions diff�rentes !
				
				\paragraph{Exemple 3}
				La cuisson d'une pizza se fait sur une feuille d'aluminium que l'on peut saisir � mains nues et non sur une plaque de m�tal.
				En effet, la capacit� thermique de la feuille d'aluminium, c'est-�-dire la capacit� d'un syst�me � stocker l'�nergie sous forme thermique, est moindre.
				On peut �galement cit� l'exemple du sauna: humidifier l'air d'un sauna augmente sa capacit� thermique, ce qui peut vite le rendre invivable !
				
				\paragraph{Exemple 4}
				Pour refroidir une bouteille de soda rapidement, on peut la plonger dans de l'eau froide 
				(avec de la glace par exemple car un m�lange d'eau et de glace garantit une eau constament � $\SI{0}{\degC}$).
				Les �changes de chaleurs se feront alors avec l'eau plut�t qu'avec de l'air, ce qui ira plus vite.
				
				\paragraph{Exemple 5}
				Une cocotte-minute acc�l�re le chauffage des aliments gr�ce � son milieu �tanche permettant une augmentation de la pression et donc une augmentation de la temp�rature d'�bullition.
				($ P \nearrow \Rightarrow T_{\text{�bullition}} \nearrow $)
				
	\section{\'E{}tat de la mati�re}

		\subsection{Le corps pur}
	
				\paragraph{D�finition}
				Un \emph{corps pur} est un constituant unique caract�ris� par une formule chimique d�finie.
				
				\paragraph{Exemples}
				\[ 
				 	He_{(g)}, 
				 	Fe_{(s)},
				 	{NH_3}_{(l)},
				 	{O_{2}}_{(g)}, ...
				\]
				Souvent on travaille avec des m�langes (comme l'air)
		
		\subsection{Les diff�rents �tats d'un corps pur}
		
			\subsubsection{Phase solide}
				\begin{itemize}
					\item \'E{}tat compact dont le mod�le id�al est la cristal parfait
					\item Distance interparticules faibles et interactions fortes $ \Rightarrow $ coh�sion et rigidit�
					\item Les solides ont un volume propre et ne peuvent pas s'�couler
					\item Quasi-incompressible et peu dilatable
				\end{itemize}
			
			\subsubsection{Phase liquide}
				\begin{itemize}
					\item Particules sont en contact mais les int�ractions moins fortes $ \Rightarrow $ mouvement possible
					\item Ils ont un volume propre mais prennent la forme d'un r�cipient
					\item Quasi-compressible et peu dilatable
				\end{itemize}
			
			\subsubsection{Phase gazeuse}
				\begin{itemize}
					\item C'est l'�tat dispers� de la mati�re. Les particules sont quasi-libres
					\item Les particules sont soumises � l'agitation thermique
					\item Peuvent se dilater ou se comprimer facilement
				\end{itemize}
		
		\subsection{L'�tat condens�}
		
				\paragraph{D�finition} \emph{\'E{}tat condens�}:\\
						Phase solide ou phase liquide quasi-incompressible.
				
				\paragraph{N.B.} Les solides et les liquides ont leur volume qui varie peu avec la temp�rature et quasi-pas avec la pression.
				
				\paragraph{Caract�ristiques}
				
				\begin{itemize}
					\item[$\bullet$] Masse volumique:				  
					    \[
					    \rho = \frac{m}{V}
					    \]
				
					Avec:
				
						\subitem $ \rho $ : masse volumique en $ \SI{}{kg.m^{-3}} ( \rho_{\mathrm{eau}} = \SI{10^{3}}{kg.m^{-3}} $ )
						\subitem $ m $ : masse en \SI{}{kg}
						\subitem $ V $ : volumue en \SI{}{m^3}
				
					\item[$\bullet$] Densit�:				  
					    \[
					    d = \frac{\rho}{\rho_{\mathrm{eau}}}
					    \]
				
					Avec:
						\subitem $ d $ : densit� (rapport sans unit�)
						\subitem $ \rho $ : masse volumique en $ \SI{}{kg.m^{-3}} $
						\subitem $ \rho_{\mathrm{eau}} $ : masse volumique de l'eau ($\SI{10^{3}}{kg.m^{-3}}$)
				\end{itemize}
				
			\subsection{\'E{}tat fluide}
			
				\paragraph{D�finition} \emph{\'E{}tat Fluide}:\\
					Gaz (compressible) ou liquide (incompressible).
				
				\paragraph{N.B.}	$ \rho_{air} \approx \SI{1.3}{kg.m^{-3}} (\rho_{gaz} $ typiquement $ \SI{1}{kg.m^{-3}}) $
				
				\paragraph{Quantit� molaire}
									
				\begin{itemize}
					\item Une mole de gaz contient $ N_A = \SI{6,022.10^{23}}{\mbox{mol�cules}} $ % mbox est ici n�cessaire pour faire passer l'accent
					\item On a:
					    \[
					    n = \frac{N}{N_A}
					    \]
					Avec:
						\subitem $ n $ : Quantit� molaire
						\subitem $ N $ : Nombre de mol�cules
						\subitem $ N_A $ : Nombre d'Avogadro
					\item On consid�re usuellement le volume molaire d'un gaz parfait �:
						\subitem $ P = \SI{10^{5}}{Pa} $ ( soit $ \SI{1}{\mbox{bar}} $)
						\subitem $ T = \SI{25}{\degC} $
						\subitem $ V_m = \SI{22.4}{L.mol^{-1}} $ ( Volume molaire dans les conditions usuelles de temp�rature et de pression )
						
						En effet, on a $ \displaystyle{PV = nRT \Rightarrow V_m = \frac{RT}{P} = \SI{22.7}{L.mol^{-1}}} $
				\end{itemize}
				
	\section{\'E{}chelles d'�tudes}

				\begin{itemize}
					\item \'E{}chelle microscopique ou mol�culaire:
					
					On a $ \SI{1}{L} \rightarrow 10^{22} $ particules $ \rightarrow $ Complexe � �tudier (trop d'�quations et de conditions initiales)
					\item \'E{}chelle macroscopique:
					
					ex: Pression et temp�rature dans une seringue
					\item \'E{}chelle m�soscopique:
					
					ex: Temp�rature dans l'amphith��tre: $ T(z, t) $
				\end{itemize}

	\section{Grandeurs thermodynamiques}

		\subsection{Vocabulaire}
		
				Un \emph{syst�me thermodynamique} est d�limit� par une surface ferm�e $ \Sigma $ (r�elle ou fictive). � travers $ \Sigma $ se produisent des transferts d'�nergie ou de mati�re.

				\emph{Syst�me ferm�}: Aucun transfert de mati�re ne peut avoir lieu (ex: ballon de foot)
				
				\emph{Syst�me ouvert}: Transfert de mati�re possible (ex: seringue)
				
				\emph{Syst�me adiabatique}: Pas de transfert thermique (ex: thermos)
				
				\emph{Syst�me isol�}: Pas de transfert d'�nergie ou de mati�re.
		
		\subsection{\'E{}quilibre thermodynamique}
		
				\paragraph{D�finition}
				Dans un syst�me thermodynamique � l'�quilibre, les diff�rentes grandeurs d�finies � l'�chelle m�soscopique sont constantes.
				
				\paragraph{Exemple}
				Champs de pression dans un fluide.
				
				\paragraph{Propri�t�}
				L'exp�rience prouve que tout syst�me isol� tend vers un �tat d'�quilibre.
		
		\subsection{\'E{}quations d'�tat}
		
			\subsubsection{Param�tres d'�tat}
			
				\paragraph{D�finition}
				Les grandeurs macroscopiques susceptibles d'�tre modifi�es lors d'une transformation d'un syst�me sont les \emph{param�tres d'�tat}.
			
			\subsubsection{Grandeurs intensives et grandeurs extensives}
			
				\paragraph{D�finition}
				\begin{itemize}
					\item Les \emph{grandeurs extensives} sont li�es � la quantit� de mati�re
					\item Les \emph{grandeurs intensives} ne sont pas li�es � la quantit� de mati�re
				\end{itemize}
				
				\paragraph{Exemples}
				\begin{itemize}
					\item Grandeurs extensives
						\subitem $ m $
						\subitem $ V $
						\subitem $ n $
					\item Grandeurs intensives
						\subitem $ Q $
						\subitem $ T $
						\subitem $ C (mol.L^{-1}) $
				\end{itemize}
				
				\paragraph{Astuce}
				Imaginez regrouper 2 syst�mes identiques:
				\begin{itemize}
					\item Ce qui double $ \Rightarrow $ Grandeurs extensives
					\item Ce qui ne bouge pas $ \Rightarrow $ Grandeurs intensives
				\end{itemize}
				
			\subsubsection{Notion de phase}
				
				\paragraph{D�finition}
				Une \emph{phase} correspond � toute partie d'un syst�me dont les grandeurs intensives sont des fonctions continues des variables d'espace.
				
				\paragraph{Exemple 1} (\reffig{bouteille_3_phases})
				
						\fig{bouteille_3_phases}
						{Les 3 phases d'une bouteille (air + huile + vinaigre)}

				Ici, la masse volumique est discontinue lors des changements de phase.
						
				\paragraph{Exemple 2} (\reffig{bouteille_5_phases})
						
						\fig{bouteille_5_phases}
						{Les 5 phases d'une bouteille (air + huile + vinaigre + 2 cailloux)}
				
				\paragraph{D�finition}
				La relation entre les param�tres d'�tat est l'\emph{�quation d'�tat}
				
				\paragraph{Exemples}
				
				\begin{itemize}
					\item \'E{}quation d'�tat des gaz parfaits:
					    \[
						    \boxed{PV = nRT}
					    \]
					\item \'E{}quation d'�tat du gaz de Van der Waals:
					    \[				
						    \boxed{\left( P + \frac{n^2 a}{V^2} \right) \left( V - nb \right) = nRT}
					    \]
				\end{itemize}

	\section{Le gaz parfait}

				\paragraph{D�finition}
				Le \emph{gaz parfait} est un gaz constitu� d'un ensemble d'atomes:
				\begin{itemize}
					\item Consid�r�s comme des particules ponctuelles
					\item Sans int�ractions entre eux
					\item Dont les chocs entre eux sont parfaitement �lastiques
				\end{itemize}
				
				\paragraph{Am�lioration du mod�le du gaz parfait}
				Pour le gaz de Van der Waals:
				\[
					\left[P + \frac{n^2 a}{V^2}\right] \left[V - nb\right] = n R T
				\]
				
				Avec:
				\begin{itemize}
					\item $ b $ : rend compte du volume des particules
					\item $ a $ : rend compte du fait que les atomes d'un gaz interagissent entre eux
				\end{itemize}
				
	
	\section{Dilatation et compressibilit� d'un syst�me}
				\begin{itemize}
					\item Coefficient de dilatation isobare:
					    \[
						    \boxed{\alpha = \frac{1}{V} \left( \frac{\partial V}{\partial T} \right)_P}
					    \]
					\item Coefficient de compressibilit� isotherme:
					    \[
						\boxed{\chi_{T} = - \frac{1}{V} \left( \frac{\partial V}{\partial P} \right)_T}
					    \]
					\item Coefficient de variation de pression isochore:
					    \[
						    \boxed{\beta = \frac{1}{P} \left( \frac{\partial P}{\partial T} \right)_V}
					    \]
				\end{itemize}
				\paragraph{N.B.} Ces coefficients permettent de caract�riser les variations dans un syst�me.
			
				\paragraph{Exemple}
				Pour un gaz parfait, d�terminons $ \alpha $ :
				\begin{align*}
				   PV								& = nRT \\
				   \Rightarrow V						& = \frac{nRT}{P} \\
				   \Rightarrow \left(\frac{\partial V}{\partial T} \right)_P 	& = \frac{nR}{P} \\
				   								& = \frac{V}{T} \\i
				   \Rightarrow \alpha						& = \frac{1}{V} \left(\frac{\partial V}{\partial T} \right)_P \\
				   \Rightarrow \alpha						& = \frac{1}{T} \\
				\end{align*}
				
				De la m�me mani�re, on trouve:
				\[
					\chi_T = \frac{1}{P}
				\]


\chapter{�l�ments de statique des fluides}

\section{Introduction}

		\subsection{G�n�ralit�s}
	
				\paragraph{�tat fluide}
				\begin{itemize}
					\item \'E{}tat liquide (quasi incompressible)
								\\ou
					\item \'E{}tat gazeux (fortement compressible)
				\end{itemize}
			
				\paragraph{Hypoth�ses}
				\begin{itemize}
					\item[*] On se place dans ce chapitre au niveau m�soscopique
					\item[*] Le fluide est un milieu continu
					\item[*] Le fluide est parfait, c'est-�-dire sans viscosit�
				\end{itemize}
		
		\subsection{Pression dans un fluide}
		
				\paragraph{Dans un fluide parfait} la force exerc�e par le fluide sur un solide est perpendiculaire � la surface du solide. 
				Cette force est dirig� vers l'int�rieur du solide (Voir \reffig{pressions_fluide}).
						
						\fig{pressions_fluide}
						{Direction des forces de pressions dans un fluide}
		
				On a:
				\[
					\d \vec{F}_M = P(M) \times \d S_M \times \vec{n}_M
				\]
		
				Avec:
				\begin{itemize}
					\item $ \d S_M $ Surface �l�mentaire autour de $ M $
					\item $ \vec{n}_M $ Vecteur unitaire perpendiculaire � la surface dirig� vers l'int�rieur
				\end{itemize}
	
	
	\section{Relation fondamentale de la statique des fluides}
	
				Soit $ M(x, y, z) $ un point du fluide sommet d'un cube de cot�s $ \d x, \d y, \d z $ (Voir \reffig{cube_fluide}).
						
						\fig{cube_fluide}
						{Entourage du point M}	
	
				\paragraph{\'E{}tude des forces de pression} On se place de c�t� (\reffig{cube_fluide_vue_cote}).
				
						\fig{cube_fluide_vue_cote}
						{Entourage du point M vu de c�t�}	
	
				Selon $ \vec{U}_x $ :
				\begin{eqnarray*}
					\d f_x	& = &	P(x, y, z) \d y \d z - P(x + \d x, y, z) \d y \d z \\
					\d f_x	& = & - \frac{\partial P}{\partial x} \d x \d y \d z
				\end{eqnarray*}
	
				De la m�me mani�re, on trouve:
				\begin{eqnarray*}
					\d f_y	& = &	- \frac{\partial P}{\partial y} \d x \d y \d z \\
					\d f_z	& = & - \frac{\partial P}{\partial z} \d x \d y \d z
				\end{eqnarray*}
	
				\paragraph{Bilan} Avec $ \vec{f} = $  r�sultante des forces de pression agissant sur le cube, on a:
				\begin{eqnarray*}
					\vec{f}	&	=	&	\underbrace{\left[	- \frac{\partial P}{\partial x} \vec{U}_x %
										- \frac{\partial P}{\partial y} \vec{U}_y %
										- \frac{\partial P}{\partial z} \vec{U}_z %
																		\right]}_{- \vec{grad}[P]} \d x \d y \d z\\
					\vec{f}	&	=	&	- \vec{grad}[P] \times \d x \d y \d z
				\end{eqnarray*}
	
				Or le cube de fluide est immobile. Donc:
				\[
					m\vec{a} = \vec{f} + \vec{P} = \vec{0}
				\]
				
				Or
				\[
					\vec{P} = - m g \vec{U_z} \mbox{    avec } n = \rho(M) \d x \d y \d z
				\]
				
				Car $ m = \rho \times \d \tau_M =  \rho(M) \d x \d y \d z $
				
				On a donc
				
				\begin{eqnarray*}
					- \vec{grad}[P] \times \d x \d y \d z - \rho(M) \times \d x \d y \d z \times g \times \vec{U_z}	&	=	&	0 \\
					\vec{grad}[P]																																										&	=	&	- \rho(M) g \vec{U_z} \\
																																																					&	=	& + \rho \vec{g}
				\end{eqnarray*}
	
				\paragraph{Autre �criture} (\reffig{direction_z})
				
						\fig{direction_z}
						{\'E{}critures diff�rentes selon le contexte d'�tude}
			
	\section{Statique des fluides homog�nes et incompressibles}
		
		\subsection{Pr�sentation}
		
				Un fluide \emph{incompressible} $ \Rightarrow \rho = \mbox{constante} $
				
				On a donc:
				\[
					\frac{\d P}{\d z} = - \rho g \Rightarrow P(z) - P(z_0) = - \rho g ( z - z_0 )
				\]
				
				Attention! On a $ - g $ ou $ + g $ en fonction du contexte d'�tude (ici, le contexte d'�tude est celui \reffig{direction_vecteur_g})
				
						\fig{direction_vecteur_g}
						{Direction du vecteur $\vec{g}$}
				
		\subsection{Exemples}
		
			\subsubsection{Pression sous-marine}
			
				On se place dans le contexte suivant (\reffig{pression_sous_marine}).
				
						\fig{pression_sous_marine}
						{Contexte d'�tude de la pression sous-marine}
			
				\begin{eqnarray*}
					\frac{\d P}{\d z}		&	=	&	+ \rho g \\
					P(z) - P(z = 0) 		&	= &	\rho g z \\
					P(z)								&	= &	P_{\mbox{atm}} + \rho g z
				\end{eqnarray*}
				
				\paragraph{Question} A quelle pression r�siste une montre �tendue � $ \SI{50}{m} $ ?
				
				Ici on a:
			
				\[
			  	\left \{
			   	\begin{array}{r c l}
			    	z  		& = 			& \SI{50}{m} \\
			      \rho	& = 			& \SI{10^3}{kg.m^3} \\
			      g			& \approx & \SI{10}{m.s^{-2}}
			   \end{array}
			   \right . \\
			   P(z=50) \approx 10^5 + 5.10^5 \approx \SI{6}{bars}
				\]
			
				\paragraph{Question} Comment attacher un bateau dans un port ?
				
						\fig{pression_sous_marine_ancre}
						{Pression sous-marine exerc�e sur un pneu plein de b�ton}
						
				On voit (\reffig{pression_sous_marine_ancre}) que l'on a:		
				\[
					F = \SI{2.10^5}{N} \Longleftrightarrow \SI{20}{tonnes}
				\]
			
			\subsubsection{Manom�tre � air libre}
						
						\fig{manometre_a_air_libre}
						{Manom�tre � air libre}
						
				On observe (\reffig{manometre_a_air_libre}):
				\begin{eqnarray*}
					\rho		&	=	&	\mbox{constante} \mbox{ (car le fluide est incompressible)}\\
					P_A		&	=	&	P	\\
					P_B		&	=	&	P_{atm}	\\
					P_A		&	=	&	P_B + \rho g h	\\
					\Rightarrow P &	=	&	P_{atm} + \rho g h
				\end{eqnarray*}
			
			
			\subsubsection{Barom�tre � mercure}
						
						\fig{barometre_a_mercure}
						{Barom�tre � mercure}
						
				On observe (\reffig{barometre_a_mercure}):		
				\begin{eqnarray*}
					P_C	&	\approx	&	0 \mbox{ (quasi vide au dessus de C)} \\
					P_A	&	=				&	P_{\mbox{atm}}
				\end{eqnarray*}
		
				$ A $ et $ B $: m�me fluide et m�me altitude $ \Rightarrow P_B = P_A $
				
				$ A $ et $ C $: m�me fluide $ \Rightarrow P_B = P_C + \rho g h $
				
				$ \Longrightarrow P{\mbox{atm}} \approx \rho g h $ et en mesurant $ h $ on mesure $ P_{\mbox{atm}} $
			
				\paragraph{Application num�rique} $ P_{\mbox{atm}} = \SI{101325}{Pa} $ correspond � $ h = \SI{760}{mm} $ de mercure
				
				\paragraph{Question} Pourquoi du mercure et pas de l'eau ?
				
				La raison est que:
				\begin{eqnarray*}
					\rho_{Hg}					&	=	&	\SI{13.6 \times 10^3}{kg.m^{-3}} \\
					\rho_{\mbox{eau}} &	= &	\SI{10^3}{kg.m^{-3}} \Rightarrow \mbox{ Si le montage utilisait de l'eau alors } h' \approx \SI{10}{m}
				\end{eqnarray*}
			
				Cette exp�rience a �t� r�alis� par \emph{Torriceli} en 1643.
				
	\section{Fluides homog�nes et compressibles}
		
		\subsection{Pr�sentation d'un mod�le de l'atmosph�re}
		
				Ici, on cherche � d�terminer $ P(z), \rho(z), T(z), ... $
				
				\paragraph{Mod�le de l'atmosph�re isotherme} hypoth�ses:
				\begin{itemize}
					\item L'air se comporte comme un gaz parfait de masse molaire $\SI{29}{g.mol^{-1}} $
					\item $ \vec{g} $ est uniforme $ \displaystyle{\left( \vec{g} = \vec{g_0} \times \frac{R^2_T}{R^2_T + z^2} \right)} $ et $ (z << R_1) $
					\item La temp�rature est constante et vaut $ T_0 $
				\end{itemize}
		
		
		
		\subsection{Altitude caract�ristique du mod�le}
		
				Premi�rement, d�terminons $P(B)$ et $\rho(z)$
				
				On a:
				\[
					\frac{\d P}{\d z} = -\rho(z) g
				\]
		
				Or:
				\begin{itemize}
					\item[*]									$ \displaystyle{\rho = \frac{m}{V} = \frac{nM}{V}} $
					\item[*] 									$ \displaystyle{PV = nRT \Rightarrow \frac{n}{V} = \frac{P}{RT_0}} $
					\item[$\Longrightarrow$]	$ \displaystyle{\rho(z) = \frac{MP(z)}{RT_0} \Rightarrow \frac{\d P}{\d z} = - \frac{Mg}{RT_0} P(z)} $
				\end{itemize}
		
				D'o�:
				\[
					\Rightarrow \frac{\d P}{\d z} = - \frac{Mg}{RT_0} P(z)
				\]
		
				Soit
				\begin{eqnarray*}
					\frac{\d P}{P}						 												&	=	&	- \frac{Mg}{RT_0} \d z \\
					\d\left[ \ln (P) \right] 													&	=	&	- \frac{Mg}{RT_0} \d z \\
					\int_{z=0}^z \left(\d\ln\left[(P)\right] \right)	&	=	&	\int_{z=0}^z -\frac{Mg}{RT_0} \d z \\
					\ln\left[P(z)\right] - \ln\left[P(0)\right]				&	=	&	-\frac{Mg}{RT_0} (z - 0) \\
					\ln\left[\frac{P(z)}{P(0)}\right]									&	=	&	-\frac{Mg}{RT_0} z \\
					P(z)																							&	=	&	P_0 \e{-\frac{Mg}{RT_0} z} \\
					P(z)																							&	=	&	P_0 \e{- \frac{z}{H}} \mbox{ avec } H = \frac{R T_0}{Mg}
				\end{eqnarray*}
		
				Puis:
				\begin{eqnarray*}
					\rho(z)									 													&	=	&	\frac{MP(z)}{R T_0} \\
					\rho(z)									 													&	=	&	\underbrace{\frac{MP_0}{R T_0}}_{\rho_0} \times \e{- \frac{z}{H}} \\
					\rho(z)									 													&	=	&	\rho_0 \e{- \frac{z}{H}} \\
				\end{eqnarray*}
		
				\paragraph{N.B.} \reffig{courbe_altitude_caracteristique}
				
						\fig{courbe_altitude_caracteristique}
						{Courbe de pression en fonction de l'altitude}	
		
				$ H = $ altitude caract�ristique
		
				A.N. 
				\[
					H = \frac{8.31 \times 293}{29.10^{-3} \times 9.81} \approx \SI{8}{km}
				\]
				
				Ici, apr�s 3H, il n'y a presque plus de pression: on atteint les limites du mod�le.
				
				D�terminons h tel que si $ z \leq h, P(z) \geq \frac{99}{100} P_0 $ 
				
				On a: 
				\[
					P(h) = P_0 \e{-\frac{h}{H}} = \frac{99}{100} P_0
				\]
				
				D'o�
				\[
					-\frac{h}{H} = \ln\left[\frac{99}{100}\right] 
					\Rightarrow h = H \ln\left[\frac{100}{99}\right] 
					\Rightarrow h \approx \SI{80}{m}
				\]
				\paragraph{N.B.} Le mod�le $ T = T_0 $ est peu r�aliste, on peut utiliser $ T(z) = T_0 \textcolor{red}{( 1 - ???)} $
		
	\section{Pouss�e d'Archim�de}
		
				\paragraph{D�finition} On appelle \emph{pouss�e d'Archim�de} la r�sultante des forces de pression qu'exerce un fluide sur un corps immerg�.
				
				\paragraph{Illustration} Soit une canette de coca immerg�e dans l'eau (\reffig{cannette_immergee})
				
						\fig{cannette_immergee}
						{Canette de coca immerg�e dans l'eau}	
	
				\paragraph{Question} Calculer la r�sultante des forces de pression.
				
				\begin{itemize}
					\item[$\bullet$] R�sultante sur la surface lat�rale $ = \vec{0} $ (\reffig{cannette_immergee_vue_dessus})
					
						\fig{cannette_immergee_vue_dessus}
						{Canette de coca immerg�e dans l'eau (vue du dessus)}
						
					\item[$\bullet$] Base sup�rieure:
													\[
														P(z_0 + h) \times S \times (- \vec{U}_z)
													\]
					\item[$\bullet$] Base inf�rieur:
														\[
															P(z_0) \times S \times (+ \vec{U}_z)
														\]
				\end{itemize}
	
	
	
	
				Donc, 
				\begin{eqnarray*}
					\vec{\Pi}_A	&	=	&	P(z_0) S \vec{U}_z - P(z_0 + h) S \vec{U}_z \\
											&	=	&	\underbrace{\left[ P(z_0) - P(z_0 + h) \right]}_{\rho g h} S \vec{U}_z \\
											& = & \rho \times \underbrace{hS}_{\mbox{Volume}} \times g \vec{U}_z
				\end{eqnarray*}
				
				\paragraph{Th�or�me d'Archim�de} Un solide immerg� dans un ou plusieurs fluides subit une force �gale et oppos�e au poids des fluides d�plac�s, c'est la \emph{pouss�e d'Archim�de}.
				Cette force s'applique au centre d'inertie des fluides d�plac�s.			


\chapter{1\ier~Principe de la Thermodynamique}

\section{Pr�sentation}
	
				On �tudie dans ce chapitre l'�volution d'un syst�me d'un �tat d'�quilibre � un autre.

				\paragraph{Exemple}
				\begin{itemize}
					\item Compression d'un gaz dans un piston
					\item Chauffage d'une casserole d'eau
				\end{itemize}
		
				\paragraph{Les transformations} peuvent �tre:
				\begin{itemize}
					\item Brusques ou brutales, les param�tres ne sont pas d�finis entre les 2 �tats d'�quilibre. Ces transformation sont n�cessairement \emph{irr�versibles}.
					\item Lentes, on est en permanence dans un �tat d'�quilibre. On parle de \emph{transformations quasistatiques}. 
								Une transformation quasistatique est \emph{r�versible} si on passe par les m�mes �tats d'�quilibre dans un sens de la transformation ou dans l'autre.
								Cela suppose l'absence de \emph{ph�nom�ne dissipatif}.
				\end{itemize}
		
				\paragraph{Ph�nom�nes dissipatifs}:
				\begin{itemize}
					\item Frottements secs ou fluide, inelasticit�
					\item Diffusion de mati�re, de chaleur, de quantit� de mouvement (viscosit�).
					\item Inhomog�n�it� du syst�me (ex: pression, temp�rature)
				\end{itemize}

	\section{�nergie d'un syst�me}
				\paragraph{L'�nergie totale} d'un syst�me est E tel que:
				\[
					E = U + Ec_{\mbox{macro}} + Ep_{\mbox{ext}}
				\]
				Avec:
				\begin{itemize}
					\item $ U $				\'E{}nergie interne ($ = Ec_{\mbox{agitation}} + Ep_{\mbox{int}} $)
					\item $ Ec_{\mbox{macro}} $	\'E{}nergie cin�tique li�e au mouvement d'ensemble
					\item $ Ep_{\mbox{ext}} $	\'E{}nergie potentielle du syst�me par rapport � l'ext�rieur
				\end{itemize}
				
				\paragraph{N.B.} Exemple d'un ballon de foot rempli d'air et de masse $m_B$ (\reffig{ballon_de_foot})
						
						\fig{ballon_de_foot}
						{\'E{}nergie du syst�me "Ballon de foot"}
										
				\[
					Ec_{\mbox{macro}} = \frac{1}{2} m_B v_B^2
				\]
				\[
					Ep_{\mbox{ext}} = m g z_B + K
				\]
				\[
					U = Ec_{\mbox{agitation}} + Ep_{\mbox{int}}
				\]
				
				Avec:
				\begin{itemize}
					\item $ Ep_{\mbox{int}} $ 	\'E{}nergie potentielle d'agitation entre les particules
					\item $Ec_{\mbox{agitation}} $ 	\'E{}nergie cin�tique des particules en mouvement � cause de l'agitation thermique
				\end{itemize}
				
				Si on a un gaz parfait, on a:
				\[
					U = Ec_{\mbox{agitation}}
				\]
				
				\paragraph{Propri�t�s de l'�nergie interne U}
				\begin{itemize}
					\item $ U $ est une grandeur extensive, c'est-�-dire $ \displaystyle{U_{\Sigma_1 \cup \Sigma_2} = U_{\Sigma_1} + U_{\Sigma_2}} $
					\item $ U $ est une fonction d'�tat c'est-�-dire que $ U $ ne d�pend que des param�tres d'�tat
				\end{itemize}
				
				
				

	\section{Transfert d'�nergie}

	   \subsection{Transferts thermiques}
	   	
	   	Il y a 3 modes de transferts thermiques:
	   	\begin{itemize}
	   		\item Par \emph{conduction} (ex: louche en argent dans la soupi�re)
	   		\item Par \emph{rayonnement} (ex: chaleur suppl�mentaire au soleil)
	   		\item Par \emph{convection} (ex: mouvement d'air dans une pi�ce chauff�e)
	   	\end{itemize}
	   	
	   	\paragraph{N.B.}
	   	\begin{itemize}
			\item 	On note $ Q $ la quantit� d'�nergie re�ue par le syst�me de la part de l'ext�rieur au cours de la transformation.
				Si la transformation est infinit�simale alors on notera $ \delta Q $ la quantit� de chaleur re�ue
			\item	$ Q $ peut �tre:
				\subitem $ > 0 $ (effectivement re�ue)
				
				ou
				\subitem $ < 0 $ (effectivement donn�e)
			\item 	Lors d'une transformation adiabatique:
				\subitem $ \delta Q = 0 $
				\subitem $ Q = 0 $ 
		\end{itemize}


	   \subsection{Travail des forces de pression}
	   		Consid�rons un piston de section $ S $: (Voir \reffig{piston_simple})
	   		
	   				\fig{piston_simple}
	   				{Travail des forces de pressions d'un piston}
																		
	   		\[
	   			\underset{\text{ext}\to \text{piston}}{\delta W}
	   			= 
	   			\vec{F}_{\text{ext}} \times \vec{d}l = \text{travail re�ue par le piston}
	   		\]
	   		
	   		En $ x $ on est � l'�quilibre: $ \displaystyle{E_c(x) = 0} $
	   		
	   		En $ x + \d x $ on est � l'�quilibre: $ \displaystyle{E_c(x + \d x) = 0} $
	   		
	   		\[
	   			\Delta E_c = 0 =	\underset{\text{ext}\to \text{piston}}{\delta W} +
	   								\underset{\text{gaz}\to \text{piston}}{\delta W}
	   		\]
	   		
	   		D'o�:
	   		\begin{itemize}
				\item[$\bullet$] $ \displaystyle{\underset{\text{gaz}\to \text{piston}}{\delta W} = 
									\vec{F}_{\text{ext}} \times \vec{d}l} $
				\item[$\bullet$]	$ \displaystyle{\underset{\text{gaz}\to \text{piston}}{\delta W} = 
									- \underset{\text{piston}\to \text{gaz}}{\delta W}} $
				\item[$\Rightarrow$] $ \displaystyle{\underset{\text{piston}\to \text{gaz}}{\delta W} =
										  - \vec{F}_{\text{ext}} \vec{d}l} $
			\end{itemize}
			
			On pose:
			\[
				\vec{F}_{\text{ext}} = P_{\text{ext}} \times S \vec{u}_x
			\]
			\[
				\vec{d}l = \d x \vec{u}_x
			\]
			
			Puis:
			\[
				\underset{\text{piston}\to \text{gaz}}{\delta W} = 
				- P_{\text{ext}} \underbrace{S \d x}_{\d V} \Rightarrow
				\underset{\text{piston}\to \text{gaz}}{\delta W} = 
				\delta W = - P_{\text{ext}} \d V
			\]
			
			On a donc:
			\[
				\underset{1 \to 2}{W} = \int_1^2 \delta W = \int_1^2 - P_{\text{ext}} \d V =
				\text{travail re�ue par le gaz de l'�tat 1 � l'�tat 2}
			\]
			
			\paragraph{N.B.:}
			\begin{itemize}
				\item Si $ \delta W > 0 $: le gaz est comprim�, il re�oit effectivement de l'�nergie.
				\item Si $ \delta W < 0 $: le gaz se d�tend, il a donn� effectivement de l'�nergie.
				\item Si la transformation est quasi-statique alors $ P = P_{\text{ext}} $ � tout instant 
						donc $ \delta W = - P \d V $
			\end{itemize}
	   		
	   \subsection{Autres formes de travail}
	   
	   		\paragraph{Travail �lectrique:} $ \delta W = R i^2 \d t = \frac{U^2}{R} \d t $
	   		
	   		(Voir \reffig{circuit_electrique})
	   		
	   				\fig{circuit_electrique}
	   				{Travail �lectrique sur un circuit}
	   			   		
	\section{Premier principe}
				\paragraph{\'E{}nonc�:} Pour un syst�me ferm�, le 1\ier principe est 
				un principe de conservation de l'�nergie, on a:
				\[
					\underset{1 \to 2}{\Delta U} = 
					\underset{1 \to 2}{W} + \underset{1 \to 2}{Q}
				\]
				
				\[
					\d U = \delta W + \delta Q
				\]
				
				\paragraph{Remarque:}
				\[
					U = C \times T
				\]
				\[
					\d U = C_V \times \d T
				\]
				
				\paragraph{Exemple 1}
				Soient 2 briques de capacit� thermique $ C_1 $ et $ C_2 $,
				 initialement aux temp�ratures $ T_1 $ et $ T_2 $.
				On les rassemble et on les isole thermiquement (Voir \reffig{deux_briques})
				
						\fig{deux_briques}
						{\'E{}tat initial du syst�me brique 1 + brique 2}
								
				\paragraph{Question} Trouver $ T_f $
				
				\[
					\Sigma_1 = \text{brique } 1
				\]
				\[
					\Sigma_2 = \text{brique } 2
				\]
				
				\[
					\underset{i \to f}{\Delta U_{\Sigma_1 \cup \Sigma_2}}	= 
					\underbrace{\underset{i \to f}{W_{\Sigma_1 \cup \Sigma_2}}}_{= 0} +
					\underbrace{\underset{i \to f}{Q_{\Sigma_1 \cup \Sigma_2}}}_{= 0} = 0
				\]
				
				$ U $ grandeur extensive:
				\begin{eqnarray*}
					\Delta U_{\Sigma_1 \cup \Sigma_2}
					& = 
					& \Delta U_{\Sigma_1} + \Delta U_{\Sigma_2}\\
					& =
					& C_1\left[T_f - T_1\right] + C_2\left[T_f-T_2\right]\\
					& = 
					& 0
				\end{eqnarray*}
				\[
					T_f = \frac{C_1 T_1 + C_2 T_2}{C_1 + C_2}
				\]
				
				\paragraph{Exemple 2}
				Dans la brique 1 on a mis une r�sistance $ R_0 $ 
				qui est parcourue par un courant $ I $ pendant 
				une dur�e $\tau$
				
				\paragraph{Question} Trouver $ T_f^\prime $
				
				\[
					\underset{i \to f}{\Delta U_{\Sigma_1 \cup \Sigma_2}} =
					\underbrace{
						\underset{i \to f}{Q_{\Sigma_1 \cup \Sigma_2}}
					}_{= 0} +
					\underbrace{
						\underset{i \to f}{W_{\Sigma_1 \cup \Sigma_2}}
					}_{= R_0 I^2 \times \tau}
				\]
				
				\[
					\Delta U_{\Sigma_1} + \Delta U_{\Sigma_2} =
					R_0 I^2 \times \tau
				\]
				\[
					C_1\left[T_f^\prime - T_1\right] + 
					C_2\left[T_f^\prime - T_2\right] =
					R_0 I^2 \times \tau
				\]
				\[
					\Rightarrow T_f^\prime = 
					\frac{C_1 T_1 + C_2 T_2 + R_0 I^2 \times \tau}{C_1 + C_2}
				\]

	\section{Fonction enthalpie}
				\paragraph{D�finition} L'\emph{enthalpie} $ H $ est la fonction d'�tat extensive, d�finie par:
				\[
					H = U + PV
				\]
				
				\paragraph{propri�t�} Pour une transformation isobare quasi-statique et o� seul le travail des forces de pression intervient, on a:
				\[
					\d H = \delta Q
				\]
				et
				\[
					\underset{1 \to 2}{\Delta H} = \underset{1 \to 2}{Q}
				\]
				
				\paragraph{D�monstration}:				
				\begin{eqnarray*}
					\d H 	& = &	\d U + \d (PV)\\
						& = &	\delta Q - P \d V + V \d P + P \d V \\
						& = & \delta Q + V \d P
				\end{eqnarray*}
				
				Ici
				\[
					\underbrace{\d P = 0}_{\text{isobare}} \Rightarrow \d H = \delta Q
				\]
				
				\paragraph{Propri�t�} Seconde Loi de Joule
				
				Pour un gaz parfait, $ H $ ne d�pend que de $ T: H = H(T) $
				
				\paragraph{Rappel} Premi�re Loi de Joule
				\[
					U = U(T) \text{ pour un gaz parfait}
				\]
				
				\paragraph{N.B.} On a
				\begin{eqnarray*}
					H & = & U + PV\\
					H & = & U(T) + nRT\\
					H & = & H(T)
				\end{eqnarray*}

	\section{Capacit�s thermiques}

	   \subsection{D�finitions}
				On d�finit:
				\[
					C_V = \left( \frac{\partial U}{\partial T}\right)_V
				\]
				\[
					C_P = \left( \frac{\partial H}{\partial T}\right)_P
				\]

	   \subsection{Cas des gaz parfaits}
	   			On a:
	   			\[
	   				U = U(T) \Rightarrow C_V = \frac{\d U}{\d T}
	   			\]
	   			\[
	   				H = H(T) \Rightarrow C_P = \frac{\d H}{\d T}
	   			\]
	   			
	   			On d�finit:
	   			\[
	   				\gamma = \frac{C_P}{C_V}
	   			\]
	   			
	   			Avec: $ \gamma = $ Constante $ \approx 1.4 $
	   			
	   			\paragraph{Propri�t�}
	   			\[
	   				C_P - C_V = nR
	   			\]
	   			
	   			\paragraph{D�monstration}:
	   			
	   			\begin{eqnarray*}
	   				H 				& = & U + PV\\
	   								& = &	U + nRT\\
	   				\frac{\d H}{\d T} 		& = & \frac{\d U}{\d T} + nR\\
	   				\Rightarrow C_P - C_V 	& = & nR
	   			\end{eqnarray*}
	   			
	   			\paragraph{Propri�t�}
	   			\[
	   				C_V = \frac{nR}{\gamma - 1}
	   			\]
	   			\[
	   				C_P = \frac{nR\gamma}{\gamma - 1}
	   			\]
	   			
	   			\paragraph{D�monstration}:
	   			\begin{eqnarray*}
	   				C_P - C_V 	& = & nR\\
	   				C_P 		& = & \gamma C_V\\
	   				\Longrightarrow \gamma C_V - C_V 	& = & nR\\
	   				C_V						& = & \frac{nR}{\gamma - 1}\\
	   				C_P						& = & \frac{nR\gamma}{\gamma - 1}\\
	   			\end{eqnarray*}

	   \subsection{Cas des gaz condens�es}
	   			Ici on a:
	   			\[
	   				C_P \approx C_V = C
	   			\]
	   			
	\section{Lois de Laplace}
				Pour un gaz parfait, lors d'une transformation quasi-statique, adiabatique o� seul le travail des forces de pression intervient, on a:
				\begin{eqnarray*}
					PV^\gamma 			& = & \text{ Constante}\\
					P^{1-\gamma} T^\gamma	& = & \text{ Constante}\\
					T V^{\gamma - 1}		& = & \text{ Constante}\\
				\end{eqnarray*}
				
				\paragraph{D�monstration}
				\[
					\underbrace{\underbrace{\d U}_{C_V \d T}}_{\frac{nR \d T}{\gamma - 1}} = \underbrace{\delta Q}_{= 0} + \underbrace{\delta W}_{-P \d V}
				\]
				
				\[
					\frac{nR \d T}{\gamma - 1} = - P \d V
				\]
				
				Or
				\begin{eqnarray*}
					nRT 				& = & PV \\
					\Rightarrow n R \d T 	& = & P \d V + V \d P \\
				\end{eqnarray*}

				On a donc:
				\begin{eqnarray*}
					n R \d T 												& = & - (\gamma - 1) P \d V \\
																			& = & P \d V +V \d P \\
					P \d V \left[1 + \gamma - 1\right] + V \d P 			& = & 0\\
					\gamma P \d V + V \d P 									& = & 0\\
					\left(\gamma P \d V + V \d P\right) \times \frac{1}{PV}	& = & 0\\
					\gamma \frac{\d V}{V} + \frac{\d P}{P}					& = & 0\\
					\gamma \d \left[ \ln V\right] + \d\left[\ln P\right]	& = &	0\\
					\text{On int�gre, }\ln\left[P V^\gamma\right]			& = & \ln\left[P_0 V_0^{\gamma - 1}\right]\\
					PV^\gamma												& = & P_0 V_0^\gamma \\
					\text{Or, } P 											& = & \frac{nRT}{V}\\
					\Rightarrow TV^{\gamma - 1}								& = & T_0 V_0^{\gamma - 1}\\
					V														& = & \frac{nRT}{P}\\
					P^{1 - \gamma} T^\gamma 								& = & P_0^{1 - \gamma} T_0^\gamma
				\end{eqnarray*}


\chapter{2\ieme~Principe de la Thermodynamique}

\section{Pr�sentation}

Le 1\ier principe est le principe de conservation de l'�nergie. Cependant, il ne donne pas d'information sur le sens de la transformation

\paragraph{Causes d'irr�versibilit�s}
\begin{itemize}
	\item Ph�nom�nes de diffusion (de particules, thermiques et de quantit� de mouvements ou viscosit�)
	\item Frottement (exemple rampe de skate)
	\item In�lasticit� (exemple du ressort ou pare-choc)
\end{itemize}

\section{Fonction entropie}

  \subsection{D�finitions}

    \paragraph{Source de chaleur}
    Une \emph{source de chaleur} est d�fini comme un syst�me ferm� qui n'�change de l'�nergie avec l'ext�rieur que sous forme de transfert thermique.

    Exemples: Brique chaude dans un bain froid ou thermos ferm�.

    \paragraph{Thermostat}
    Un \emph{thermostat} est une source de chaleur particuli�re dont la temp�rature est constante.

    Exemples: M�lange eau/glace ou l'atmosph�re

    \paragraph{N.B.}
    Un thermostat �change de l'�nergie avec l'ext�rieur de fa�on \emph{r�versible}.

  \subsection{�nonc� du 2e Principe}

    \subsubsection{�nonc� sous forme int�gr�e}

Le second principe postule l'existence d'une \emph{fonction d'�tat} extensive $ S $, 
l'entropie d�finie par:
\[
 \underset{A \rightarrow B}{\Delta S} = \underset{A \rightarrow B}{S^e} + \underset{A \rightarrow B}{S^c}
\]
Avec:

\[
 \underset{A \rightarrow B}{S^e} = \int \frac{\delta Q_e}{T_e} = \text{ entropie d'�change}
\]

\begin{eqnarray*}
\underset{A \rightarrow B}{S^C} & = & \text{ entropie de cr�ation} \\
				& = & 0 \text{ si la transformation de A � B est r�versible} \\
				& > & 0 \text{ si la transformation de A � B est irr�versible} \\
\end{eqnarray*}

\paragraph{N.B.}
\begin{itemize}
	\item $ \delta Qe = \delta Q $ transfert thermique re�u (dans 9 cas sur 10, chaleur re�u ) de la part de la ou les sources ext�rieures
	\item $ Te = $ temp�rature de l'interface (Voir \reffig{melange_eau_glace})
	
				\fig{melange_eau_glace}
				{M�lange d'eau et de glace (m�lange � $ \SI{0}{\degC} $ )}
				
	\item $ [ S ] = J.K^-1 $
\end{itemize}

\subsubsection{�nonc� sous forme diff�rentielle}

On a:

\[
 dS = \delta S^e + \delta S^c
\]

Avec:
\begin{eqnarray*}
	\delta S^e 	& = & \frac{\delta Qe}{Te}\\
	\delta S^c 	& = & 0 \mbox{ si r�versible} \\
		     	& > & 0 \mbox{ si irr�versible} \\
\end{eqnarray*}

\subsubsection{Exemple}

\paragraph{�nonc�}

Soit 2 solides $ \Sigma_1 $ et $ \Sigma_2 $, de capacit�s thermiques $ C_1 $ et $ C_2 $, de temp�ratures initiales $ T_1 $ et $ T_2 $.
On les assemble en les isolant (Voir \reffig{deux_briques})

		\fig{deux_briques}
		{Deux briques isol�s thermiquement et mis en contact}

Q) Montrons que $ S^c > 0 $ pour $ T_1 \neq  T_2 $

On a vu: 
\[
 T_f = \frac{C_1 T_1 + C_2 T_2}{C_1 + C_2}
\]

\paragraph{Calcul de $\Delta S_{\Sigma_1}$ ?}
�tat initial de $ \Sigma_1 = T_1 $
�tat final de $ \Sigma_1 = T_f $

On imagine une transformation fictive faisant passer $ \Sigma_1 $ de l'�tat initial ( $ T_1 $ ) � l'�tat final ( $ T_f $ ) de fa�on \emph{r�versible}.

On a: 
\[
	dS = \delta S^e + \underbrace{\delta S^C}_{ = 0 \text{ car r�versible}}
\]

Or
\begin{eqnarray*}
	dU_{\Sigma_1}	& = & C_1 \d T \\
	   			& = & \delta Q + \underbrace{\delta W}{ = 0}
\end{eqnarray*}

D'o�:
\[
 \Rightarrow \boxed{\delta Q = C_1 \d T}
\]

On a donc:
\[
	\d S = \delta S^e = \frac{C_1 \d T}{Te}
\]

Or la transformation est r�versible:
\[
	\Rightarrow T_e = T \Rightarrow \d S = C_1 \frac{\d T}{T}
\]

$ S $ est fonction d'�tat
\[
	\boxed{\Delta S_{\Sigma_1} = C_1 \ln \left(\frac{T_f}{T_1}\right)}
\]

\paragraph{Calcul de $\Delta S_{\Sigma_2}$ ?}
De la m�me mani�re on trouve:
\[
	\boxed{\Delta S_{\Sigma_2} = C_2 \ln\left(\frac{T_f}{T_2}\right)}
\]

\paragraph{Calcul de $S^C$ ?}
On a:
\[
	\Delta S_{\Sigma_1 \cup \Sigma_2} = 
	\underbrace{S^e_{\Sigma_1 \cup \Sigma_2}}_{= 0 \text{ car $\Sigma_1 \cup \Sigma_2 $ isol�}} + 
	\underbrace{S^C_{\Sigma_1 \cup \Sigma_2}}_{S^C}
\]

On a donc:
\[
	\boxed{S^C = C_1 \ln \left(\frac{T_f}{T_1}\right) - C_2 \ln \left(\frac{T_f}{T_2}\right)}
\]


\paragraph{Montrons que $ S^C > 0 $ si $ T_1 \neq T_2 $}
~\newline
Supposons $ C_1 = C_2 = C$ :
\[
	\Rightarrow T_f = \frac{T_1 + T_2}{2}
\]

\[
	\boxed{S^C = C \ln \left[ \frac{(T_1 + T_2)^2}{4 \pi T_f} \right]}
\]

On a aussi:

\[
 S^C = C \ln \left[ \frac{\left( 1 + \frac{T_2}{T_1}\right)^2}{4 \frac{T_1}{T_2}} \right]
\]

�tudions: $ f(x) = \frac{(1 + x)^2}{x} $ pour $ x \in ] 0 ; +\infty [ $

On a: 
\begin{eqnarray*}
	f'(x) 	& = & \frac{2x (1 + x) - (1 + x^2)}{x^2}\\
		& = & \frac{(1 + x)(x - 1)}{x^2}\\
		& = & \frac{x^2 - 1}{x^2}
\end{eqnarray*}



\begin{tikzpicture}
	\tkzTabInit{$x$/1,$f^\prime$/1,$f$/2}{$0$, $1$, $+\infty$}
	\tkzTabLine{,-, ,+}
	\tkzTabVar{+/$$,-/$$,+/$$}
\end{tikzpicture}

On a donc:

\begin{tikzpicture}
	\tkzTabInit{$\frac{T_1}{T_2}$/1, $S^C$/2}{$0$, $1$, $+\infty$}
	\tkzTabVar{+/$$,-/$0$,+/$$}
\end{tikzpicture}

\begin{eqnarray*}
	S^C 	& = & 0 \text{ si } T_1 = T_2 \text{ r�versible}\\
		& > & 0 \text{ si } T_1 \neq T_2 \text{ irr�versible}\\
\end{eqnarray*}

\subsubsection{Syst�me isol�}

Si le syst�me est isol� alors $\delta S^e = 0 \Rightarrow \d S = \delta S = \delta S^C \geq 0$. Dans un syst�me isol� l'entropie ne fait que cro�tre.

\section{Identit� thermodynamique}
\subsection{\'E{}nonc�}
On a pour une transformation r�versible:
\[
	\boxed{\d U = T \d S - P \d V}
\]

\paragraph{N.B.} Formule utilisable tout le temps

\paragraph{Justification de l'expression} On a:
\[
	\d U = \delta Q + \delta W
\]

Or
\[
	\d S = \delta S^e = \frac{\delta Q}{T}
\]

R�versible $ \Rightarrow T_e = T \Rightarrow \d S = \frac{\delta Q}{T} \Rightarrow \boxed{\delta Q = T \d S}$

\[
	\delta W = P \d V \text{ ici seulement les forces de pressions}
\]

\subsection{Autre formulation}

Avec:
\begin{eqnarray*}
	H 	& = & U + PV\\
	\d H 	& = & \d U + P \d V + V \d P \\
		& = & T \d S - P \d V + P \d V + V \d P\\
		& = & T \d S + V \d P
\end{eqnarray*}

\[
	\boxed{\d H = T \d S + V \d P}
\]


\subsection{Application aux solides}
Soit une transformation quelconque faisant passer de l'�tat $(T_A)$ � l'�tat $(T_B)$.

On a
\begin{eqnarray*}
	 \d U	& = & C\d T\\
	 	& = & T \d S - P\d V \\
	 	& = & T\d S - 0 \\
	\d S = C \frac{dT}{T}
\end{eqnarray*}


\[
	\boxed{\underset{A \rightarrow B}{\Delta S} = C \ln \left( \frac{T_B}{T_A} \right)}
\]

\subsection{Application aux gaz parfaits}

\paragraph{Expression de $ \Delta S_{A \to B} $ en variables $ (T, V) $}

\[
	\d U = \underbrace{C_V \d T}_{ \frac{nR}{\gamma - 1} \d T} = T\d S - P\d V
\]

\[
	\d S = \frac{nR}{\gamma - 1} \frac{\d T}{T} + \frac{P \d V}{T}
\]

Or
\[
	\frac{P}{T} = \frac{nR}{V}
\]

Donc,
\[
	\d S = \frac{nR}{\gamma - 1} \frac{\d T}{T} + nR \frac{\d V}{V}
\]

\[
	\boxed{\underset{A \to B}{\Delta S} = 
	\frac{nR}{\gamma - 1} \ln\left(\frac{T_B V_B^{\gamma - 1}}{T_A V_A^{\gamma - 1}}\right)}
\]

Expression de $\underset{A \to B}{\Delta S}$ en variable $ (P, V) $ ou $ (T, P) $


\chapter{Machines Thermiques}

	\section{Introduction}
	
				\begin{itemize}
					\item Moteurs thermiques (essence, diesel, ...)
					\item Frigo, climatisation
					\item Pompe � chaleur
				\end{itemize}
				
				\paragraph{Principe de fonctionnement d'un frigo} (Voir \reffig{frigo})
						
						\fig{frigo}
						{Fonctionnement de la machine thermique "Frigo"}				

	\section{Mod�lisation}
		\subsection{D�finitions}
		
				\begin{itemize}
					\item Source de chaleur et thermostat: cf. Second Principe
					\item Les sources de chaleur seront ici suppos�es �tre des thermostats
						\[
							\Rightarrow \Delta S_{\text{source}} = \frac{Q_{\text{source}}}{T_{\text{source}}} \mbox{ Avec: } S^C_{\text{source}} = 0
						\]
				\end{itemize}
				
				\paragraph{Source m�conique:} Tout syst�me qui �change de l'�nergie m�canique sans transfert thermique
				
				\paragraph{D�finition} Une \emph{machine thermique} est un syst�me dans lequel un fluide d�crit un cycle

		\subsection{Machine ditherme}
				Une \emph{machine ditherme} est une machine qui �change de l'�nergie thermique avec 2 sources de chaleur (Voir \reffig{machine_ditherme}):
				\begin{itemize}
					\item Source chaude � $ T_C $
					\item Source froide � $ T_f $
				\end{itemize}
				Avec $ T_f < T_C $
						
						\fig{machine_ditherme}
						{Fonctionnement d'une machine dithermique}
								
				\begin{itemize}
					\item $ Q_C = $ Chaleur re�ue par le fluide de la part de la source \emph{chaude}
					\item $ Q_f = $ Chaleur re�ue par le fluide de la part de la source \emph{froide}
					\item $ W = $ Travail re�u par le fluide
				\end{itemize}
				
				\paragraph{Exemple du frigo}
				\[
				\accolades{ 	Q_C & < & 0\\
						Q_f & > & 0\\
						W & > & 0}
				\]
				
				\paragraph{Bilan thermodynamique}:
				
				1\ier principe:
				\[
					\Delta U_{\text{cycle}} = Q_C + Q_f + W \underbrace{ = 0}_{\text{car U fonction d'�tat}}
				\]
				\[
					\boxed{Q_C + Q_f + W = 0}
				\]
				
				Second principe:
				\[
					\Delta S_{\text{cycle}} \underbrace{= 0}_{\text{car S fonction d'�tat}} = S^e_{\text{chaud}} + S^e_{\text{froid}} + S^C
				\]
				or 
				\[
					S^e_{\text{chaud}} = \frac{Q_C}{T_C}
				\]
				\[
					S^e_{\text{froid}} = \frac{Q_f}{T_f}
				\]
				
				On a donc:
				\[
					\frac{Q_C}{T_C} + \frac{Q_f}{T_f} + S^C = 0
				\]
				\[
					\Rightarrow \boxed{\frac{Q_C}{T_C} + \frac{Q_f}{T_f} \leq 0}
				\]
				
				\paragraph{Diagramme de Raveau} C'est le diagramme des zones de fonctionnement dans le plan $ Q_C, Q_f $ (Voir \reffig{plan_Qc_Qf})
				
						\fig{plan_Qc_Qf}
						{Plan d'�tude d'un diagramme de Raveau}
																					
				On a:
				\[
					\frac{Q_C}{T_C} + \frac{Q_f}{T_f} \leq 0
				\]
				\[
					\Rightarrow \boxed{Q_C \leq - \frac{T_C}{T_f} Q_f}
				\]
				
				On a un fonctionnement moteur si $ W < 0 $, Or:
				\[
					W = - Q_C - Q_f
				\]
				\[
					\Rightarrow \text{Fonctionnement moteur si } -Q_C - Q_f < 0 \Rightarrow \boxed{Q_C > - Q_f}
				\]
				On obtient le diagramme \reffig{diagramme_de_raveau}
				
						\fig{diagramme_de_raveau}
						{Diagramme de Raveau}
				
				Pour avoir un cycle moteur, il faut:
				\begin{itemize}
					\item[$\bullet$] $ Q_C > 0 $ : Le fluide \emph{re�oit} effectivement de l'�nergie de la source \emph{chaude}
					\item[$\bullet$] $ Q_f < 0 $ : Le fluide \emph{donne} effectivement de l'�nergie � la source \emph{froide}
				\end{itemize}
				
		\subsection{\'E{}tude des moteurs thermiques}
				On a:
				\[
				\accolades{ 	Q_C & > & 0\\
						Q_f & < & 0\\
						W & < & 0}
				\]
				
				\paragraph{Rendement d'un moteur}
				\[
					\boxed{\eta = \frac{|W|}{Q_C} = - \frac{W}{Q_C}}
				\]
				Or ici
				\[
					- W = Q_C + Q_f
				\]
				
				Donc, 
				\[
					\eta = \frac{Q_C + Q_f}{Q_C}
				\]
				
				Donc,
				\[
					\eta = 1 + \frac{Q_f}{Q_C}
				\]
				
				Or,
				\begin{eqnarray*}
					\frac{Q_f}{T_f} + \frac{Q_C}{T_C} 	& \leq &	0\\
					\frac{Q_f}{T_f}				& \leq &	- \frac{Q_C}{T_C}\\
					\frac{Q_f}{Q_C}				& \leq &	- \frac{T_f}{T_C}
				\end{eqnarray*}
				
				\[
					\Rightarrow \boxed{\frac{Q_f}{Q_C}\leq - \frac{T_f}{T_C}}
				\]
				
				On a donc:
				\[
					\eta = 1 + \frac{Q_f}{Q_C} \leq 1 - \frac{T_f}{T_C}
				\]
				\[
					\Rightarrow \boxed{\eta \leq \frac{T_C - T_f}{T_C}} = \eta_{\text{max}}
				\]
				
				\paragraph{Exemple} Si:
				\begin{itemize}
					\item $ T_f = \SI{20}{\degC}$
					\item $ T_C = \SI{1000}{\degC} \Rightarrow \eta_{\text{max}} = \SI{77}{\%}$
				\end{itemize}
				
				\paragraph{N.B.} $\eta_{\text{max}} $ est atteint pour des cycles r�versibles (qui n'existent pas)
				
		\subsection{\'E{}tude des machines frigorifiques}
				On a:
				\[
				\accolades{ 	Q_C & < & 0\\
						Q_f & > & 0\\
						W & > & 0}
				\]
				
				\paragraph{\'E{}fficacit�}
				\[
					\boxed{e = \frac{Q_f}{W}}
				\]
				
				Or,
				\[
					W = - Q_C - Q_f
				\]
				\[
					\Rightarrow e = \frac{- Q_f}{Q_C + Q_f} = - \frac{1}{1 + \frac{Q_C}{Q_f}}
				\]
				
				Or,
				\begin{eqnarray*}
					\frac{Q_C}{T_C}			& \leq  & - \frac{Q_f}{T_f}\\
					\frac{Q_C}{Q_f} 			& \leq  & - \frac{T_C}{T_f} \\
					1 + \frac{Q_C}{Q_f}		& \leq  & 1 - \frac{T_C}{T_f} \\
					\frac{1}{1 + \frac{Q_C}{Q_f}}	& \geq & \frac{1}{1 - \frac{T_C}{T_f}} = \frac{T_f}{T_f - T_C}\\
					\Rightarrow e 			&   =    & - \frac{1}{1 + \frac{Q_C}{Q_f}} \leq \frac{T_f}{T_C - T_f}
				\end{eqnarray*}
				
				\[
					\boxed{e \leq e_{\text{max}} = \frac{T_f}{T_C - T_f}}
				\]
				
				\paragraph{N.B.} Exemple:
				\begin{itemize}
					\item $ T_C =\SI{20}{\degC} $
					\item $ T_f = \SI{0}{\degC} $
					\item $ e_{\text{max}} = \frac{273}{20} \approx 13.5$
				\end{itemize}
				
				Si $ e = 4 $, cela signifie que pour refroidir le frigo de $ \SI{5}{kJ} $ il faut payer � EDF $ \SI{1.25}{kJ} $
				
				\subsection{Pompes � chaleur}
					\subsubsection{Pr�sentation}
					
					Une \emph{pompe � chaleur} "pompe" de l'�nergie gratuite � l'ext�rieur de l'habitation (dans l'air, le sol, de l'eau ...) pour la transporter � l'int�rieur de la maison (\reffig{pompe_chaleur})
					
					\fig{pompe_chaleur}
					{Fonctionnement de la machine thermique "Pompe � chaleur"}
					
					\subsubsection{Formalisation thermodynamique}
					
					\fig{machine_ditherme_2}
					{Fonctionnement d'une machine dithermique}
					
					\begin{itemize}
						\item Source chaude = int�rieur maison
						\item Source froide = ext�rieur maison
					\end{itemize}
					
					\[
						\accolades{ 	Q_C & < & 0\\
								Q_f & > & 0\\
								W & > & 0}
					\]
				
					
					On a:
					\[
						\accolades{	W + Q_C + Q_f & = & 0\\
								\frac{Q_C}{T_C} + \frac{Q_f}{T_f} & \leq & 0}
					\]
					
					\subsubsection{Efficacit� de la Pompe � Chaleur (PAC)}
					
					On a:
					
					\[
						e = \frac{- Q_C}{W} \text{\space} (Q_C < 0)
					\]
					
					On a de plus:
					\[
						W = - Q_C - Q_f
					\]
					
					Donc,
					\[
						e = \frac{Q_C}{Q_C + Q_f} = \frac{1}{1 + \frac{Q_f}{Q_C}}
					\]
					
					Or,
					\begin{eqnarray*}
						\frac{Q_C}{T_C} + \frac{Q_f}{T_f} & \leq & 0 \\
						\frac{Q_f}{T_f} & \leq & -\frac{Q_C}{T_C}\\
						\frac{Q_f}{Q_C} & \geq & - \frac{T_f}{T_C}
					\end{eqnarray*}
					
					Donc,
					\begin{eqnarray*}
						1 + \frac{Q_f}{Q_C} & \geq & 1 - \frac{T_f}{T_C} = \frac{T_C - T_f}{T_C}\\
						e 			   & \leq & e_{\text{max}} = \frac{T_C}{T_C - T_f}
					\end{eqnarray*}
					
					\paragraph{A.N.}
					\begin{itemize}
						\item $ T_C = \SI{20}{\degC} $
						\item $ T_f = \SI{0}{\degC} $
						\item $ e_{\text{max}} = \frac{293}{20} \approx 15 $
					\end{itemize}
					
					En pratique, $ e \approx 3 \text{ � } 4 $
					
					\subsubsection{Exemple}
					
					A la construction d'une maison, on doit installer un chauffage. La puissance est de $ \SI{6}{kW} $ en moyenne pendant les six mois les plus froids.
					
					Il y a deux possibilit�s :
					\begin{itemize}
						\item Chauffage 100\% �lectrique : 
							\subitem Co�t d'installation : $ \SI{0}{\euro} $
							\subitem Co�t de maintenance : $ \SI{0}{\euro/an} $
						\item Pompe � Chaleur, avec une efficacit� $ e = 3 $:
							\subitem Co�t d'installation : $ \SI{10000}{\euro} $
							\subitem Co�t de maintenance : $ \SI{400}{\euro/an} $
					\end{itemize}
					
					\paragraph{Question} Au bout de combien de temps la Pompe � Chaleur est "rentable" ?
					
					Donn�e : $ \SI{1}{kWh} \approx 0,13\euro $
					
					\paragraph{�tude du 100\% �lectrique :}
					
					\[
						\SI{6}{kWh} \times \underbrace{24}_{\SI{24}{h/j}} \times \underbrace{30}_{\SI{30}{j/m}} 
						\times \underbrace{6}_{\SI{6}{mois}} \times 0,13 = \SI{3370}{\euro/an} 
					\]
					
					\paragraph{�tude de la Pompe � Chaleur :}
					On a ici :
					
					\[
						e = \frac{P_{\text{chauffage}}}{P_{\text{EDF}}} \Rightarrow P_{\text{EDF}} = \SI{2}{kW}
					\]
					
					\begin{itemize}
						\item Co�t total EDF : $ \SI{1120}{\euro/an} $
						\item Co�t maintenance : $ \SI{400}{\euro/an} $
						\item Co�t total : $ \SI{1520}{\euro/an} $
					\end{itemize}
					
					Il faut donc 6 ann�es et demi pour amortir la pompe � chaleur.
					
					\subsection{Cycle de Carnot}
					
					\paragraph{D�finition} Un cycle de Carnot est compos� de 2 isothermes et de 2 adiabatiques (Ce cycle est r�versible).
					
					\paragraph{N.B.} \emph{R�versible} $ \Rightarrow e $ ou $ \eta $: maximum
					
					\paragraph{Repr�sentation dans un diagramme de Clapeyron (\reffig{diagramme_de_clapeyron})}
					
					\fig{diagramme_de_clapeyron}
					{Diagramme de Clapeyron d'un cycle de Carnot}
					
					\paragraph{Repr�sentation dans un diagramme entropique (\reffig{diagramme_entropique})}
					
					\fig{diagramme_entropique}
					{Diagramme entropique}
					
					\paragraph{Diagramme de Clapeyron} C'est une repr�sentation des transformations quasi-statiques. On peux tracer les courbes selon les cas (voir \reffig{diagramme_de_clapeyron_typique})
					
					\fig{diagramme_de_clapeyron_typique}
					{Courbes d'un diagramme de Clapeyron selon les cas ($A \to B$)}



\chapter{Le corps pur diphas�}

\section{Pr�sentation}

	\subsection{Dans la vie quotidienne}
	
	\begin{itemize}
		\item Eau bouillante: $ \SI{100}{\degC} $ � $ P = \SI{1}{atm} $. On a $ T = \SI{100}{\degC} $  quel que soit la puissance de chauffage. Attention: l'eau est bouillante � $ \SI{95}{\degC} $ � $ \SI{4000}{m}$
		\item M�lange eau / glace : $ \SI{0}{\degC} $ Avec des pieds dans le m�lange la temp�rature reste � $ \SI{0}{\degC} $ tant qu'il y a des gla�ons.
		\item Au Moyen �ge: "Chauffage � bassines d'eau". Les bassines d'eau dans le sellier "chauffent" l'air qui est en dessous de $ \SI{0}{\degC} $ 
		\item Fabrication de sorbets avec une bassine plong�e dans un m�lange d'eau sal�e et de glace sal�e � $ \SI{-3}{\degC} $ (voir \reffig{sorbet_moyen_age})
		
		\fig{sorbet_moyen_age}
		{Fabrication de sorbets au Moyen-�ge}
		
		\item Temp�rature d'�bullition et de fusion diff�rent selon les corps purs
	\end{itemize}
	
	\subsection{Vocabulaire}
	
	Voir \reffig{condensation_sublimation} pour le vocabulaire des changements d'�tats de la mati�re.
	
	Attention ! Il faut faire attention aux notations : l'enthalpie et l'entropie en \emph{minuscules} ($ h(T), s(T) $) correspondent � l'enthalpie \emph{massique} et � l'entropie \emph{massique}.
	
	\fig{condensation_sublimation}
	{Les changements d'�tats de la mati�re}
	
	\subsection{Propri�t�s}
	
	A pression constante le changement d'�tat s'effectue �galement � temp�rature constante. Pour de l'eau bouillante, toute l'�nergie apport�e sert au changement d'�tat.
	
	On a donc
	<Laisser de la place ici pour plus tard>
	
\section{\'E{}tude thermodynamique}

	\subsection{Enthalpie de changement d'�tat}
	
	\paragraph{D�finition} On appelle enthalpie massique de transition de phase, � la temp�rature $ T $, l'�nergie thermique transf�rer r�versiblement pour faire passer l'unit� de masse du corps pur de la phase 1 � la phase 2.
	
	Cette grandeur s'exprime en $ \SI{}{J.kg^{-1}} $
	
	\[
		\boxed{\underset{1 \to 2}{\Delta h(T)} = \underset{1 \to 2}{L} = \underbrace{h_2 (T)}_{\text{enthalpie massique � l'�tat 2}} - \underbrace{h_1 (T)}_{{\text{enthalpie massique � l'�tat 1}}}}
	\]
	
	
	
	\paragraph{N.B.} :
	\begin{itemize}
		\item $  \underset{1 \to 2}{L} = \underset{1 \to 2}{\Delta} $ est une grandeur alg�brique, positive dans le cas d'un passage d'un �tat � un �tat plus d�sordonn�.
		\item Quelques valeurs:
	\end{itemize}
	
	\[
	\begin{tabular}{c| c c c c}
				&	$T_f (K)$	&	$L_f (kJ.kg^{-1})$	&	$T_V (K)$	&	$L_V (kJ.kg^{-1})$	\\
				\hline
		$H_2O$	&	273		&	334			&	373		&	2257			\\
		$O_2$	&	54		&	14			&	90		&	213			\\
		$NH_3$	&	198		&	452			&	258		&	1369			\\
	\end{tabular}
	\]
	
	\subsection{Entropie de changement d'�tat}
	
	Le changement d'�tat peut �tre consid�r� comme r�versible. On a donc:
	
	\[
		\boxed{
		\underset{1 \to 2}{\Delta s(T)} = 
		s_2(T) - s_1(T) =  
		\frac{{L_{1 \to 2} (T)} }{T}
		}
	\]
	
	\paragraph{Exemple} Pour l'eau:
	
	\[
		\underset{S \to L}{\Delta s(\SI{273}{K})} = \SI{1.22}{kJ.K{-1}.kg^{-1}}
	\]
	
	\[
		\underset{L \to V}{\Delta s(\SI{373}{K})} = \SI{6.05}{kJ.K{-1}.kg^{-1}}
	\]
	
	On a:
	\[
		\underset{S \to L}{\Delta s} \mbox{ et } \underset{L \to V}{\Delta s} > 0
	\]
	
	C'est normal, car le d�sordre augmente.

	\section{Diagramme de changement d'�tats}
		\subsection{Isothermes et diagrammes (P, V)}
		
		<Ins�rer diagramme>
		
		Isotherme � temp�rature telle que $T_1 < T_2 < T_3 < T_C < T_4$
		
		\paragraph{Zone 1}: de $A_0$ � $A_2$
		\begin{itemize}
			\item Fortement compressible ($P$ augmente peu avec $v$)
			\item Milieu gazeux
			\item En $A_2$ apparition de la premi�re goutte de liquide
		\end{itemize}
		
		\paragraph{Zone 2}: de $A_2$ � $A_4$
		\begin{itemize}
			\item $P$ reste constante avec $v$ diminuant. Le changement d'�tat a lieu quand on comprime le gaz qui devient liquide.
			\item On a coexistence ddes 2 �tats liquides et vapeur
			\item En $A_2 \approx 100\% $ gaz mais en $A_n \approx 100\%$ liquide.
			\item En $A_2$ apparition de la premi�re goutte de liquide
		\end{itemize}
		
		\paragraph{Zone 3}: de $A_4$ � $A_6$
		\begin{itemize}
			\item Faible compressibilit� (il faut augmenter beaucoup $P$ pour obtenir une faible diminution de $v$)
			\item On est ici $100\%$ liquide (phase condens�s)
		\end{itemize}
		
		\paragraph{N.B.} Si $T^\prime > T$ On a la m�me allure d'isotherme. 
		Mais si $T^\prime > T_C$ on n'a plus le palier de \emph{transition de phase}.
		
		\subsection{Diagramme (P, T)}
		
		<Ins�rer diagramme>
		
		Les courbes repr�sentent les changements d'�tat.
		
		\paragraph{Courbe 1}: �quilibre solide / liquide
		\begin{itemize}
			\item Pente positive tr�s raide, illimit�s vers les hautes pressions.
			\item La temp�rature de fusion varie peu avec la pression
		\end{itemize}
		
		\paragraph{Courbe 2}: �quilibre solide / vapeur
		\begin{itemize}
			\item Pente positive
		\end{itemize}
		
		\paragraph{Courbe 3}: �quilibre liquide / vapeur
		\begin{itemize}
			\item Limit� par le point C (point critique) au del� duquel on n'a plus de transition de phase.
			\item On parle de fluide \emph{hypercritique}
			\item Pente positive $\Rightarrow$ temp�rature d'�bullition augmente avec $P$
		\end{itemize}
		
		\paragraph{Point T}: Au point triple, les trois phases cohabitent.
		
		<Tableau ici>
		\[
		\begin{tabular}{c| c c c c}
					&	Point	&	Triple&	Point &	Critique\\
					\hline
		\end{tabular}
		\]
		
		\paragraph{Cas de l'eau}
		<ins�rer sch�ma>
		
		En comprimant de la glace, on la fait fondre.
		
		\paragraph{Lien entre les 2 diagrammes}
		<Diagramme>
		
		\section{Grandeurs massiques}
			\subsection{D�finition}
			
			On consid�re un �quilibre liquide / vapeur. On a:
			\[
				m = m_V + m_L
			\]
			
			Avec:
			\begin{itemize}
				\item $m$ : masse totale
				\item $m_V$ : masse de vapeur
				\item $m_L$ : masse de liquide
			\end{itemize}
			
			On d�finit:
			\[
				\boxed{x = \frac{m_V}{m} = \text{Titre massique en vapeur}}
			\]
			
			\subsection{Grandeurs massiques}
			
			Soit un m�lange liquide / vapeur � la temp�rature $T$.

			Calculons $h(x, T) =$ enthalpie massique � la temp�rature $T$ pour une proportion $x$ de vapeur.
			
			Soit une masse $m$ du m�lange, on a:
			\begin{align*}
				H(x, T)	& = m h(x, T) \\
						& = H_V(x, T) + H_L(x, T) \\
						& = m_V h_V(T) + m_L h_L(T)
			\end{align*}
			
			\begin{align*}
				h(x, T) 	& = 	\underbrace{\frac{m_V}{m}}_{x} h_V(T) + 
							\underbrace{\frac{m_L}{m}}_{\frac{m - m_V}{m} = 1 - x} h_L (T) \\
						& = x h_V(T) + (1 - x) h_L(T) \\
			\end{align*}
			
			On a donc :
			\[
				\boxed{h(x, T) = \Big[ h_V(T) - h_L(T)\Big] x + h_L(T)}
			\]
			
			\paragraph{N.B.}
			\begin{itemize}
				\item Si $ x = 0 $ ($100 \%$ liquide) $h(0, T) = h_L(T) \Rightarrow $ Logique
				\item Si $ x = 1 $ ($100 \%$ vapeur) $h(1, T) = h_V(T) \Rightarrow $ Logique
				\item On peut g�n�raliser pour $ u(x, T), \theta(x, T), s(x, T), ...$
			\end{itemize}
			


\chapter{Diffusion thermique}

\section{Loi de Fourier}

	\subsection{Introduction}
	
	\begin{itemize}
		\item La chaleur va du chaud vers le froid.
		\item Consid�rons une surface $S$ entre $t$ et $t + \d t$, un transfert thermique $ \delta Q$ a lieu.
		\item On pose 
			\[
				\boxed{\delta Q = \Phi \d t}
			\]
			O� $\Phi = $ flux thermique � travers $S$
	\end{itemize}
	
	\paragraph{N.B.}
	\begin{itemize}
		\item $\big[ \Phi\big] = J.s^{-1} = W$
		\item Sur un mur, $\Phi$ est la puissance totale traversant le mur d'un point de vue macroscopique.
		\item Au niveau microscopique, on d�finit $\vec{j_Q}$ tel que
			\[
				\boxed{\Phi = \iint_{P \in S} \vec{j_Q}(P) \d \vec{S_P}}
			\]
			Avec :
			\begin{itemize}
				\item $\vec{j_Q}= $ vecteur densit� de courant de chaleur
				\item $\big[ j_Q\big] = W m^{-2}= $ c'est une puissance surfacique.
			\end{itemize}
	\end{itemize}
	
	\subsection{Loi de Fourier}
	
	\paragraph{\'Enonc�} Au voisinage de l'�quilibre ($\vec{\text{grad}} T$ pas trop grand), on a:
	\[
		\boxed{\vec{j_Q} = - \lambda \vec{\text{grad}} T}
	\]
	Avec :
	\begin{itemize}
		\item $ \vec{j_Q} $ en $Wm^{-2}$
		\item $ \lambda $ en $W m^{-1} K^{-1} $
		\item $T$ en $K$
	\end{itemize}
	
	\paragraph{N.B.}
	\begin{itemize}
		\item $\lambda = $ conductivit� du milieu
			\subitem Si $ \lambda$ grand (ex: m�taux), milieu bon conducteur thermique
			\subitem $\lambda$ petit (ex: air, laine de verre, ...), milieu mauvais.

	\end{itemize}


\backmatter



\end{document}
